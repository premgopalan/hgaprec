\subsection{Related work}

The roots of Poisson factorization come from nonnegative matrix
factorization~\cite{Lee:1999}, where the objective function is
equivalent to a factorized Poisson likelihood.  The original NMF
update equations have been shown to be an expectation-maximisation
(EM) algorithm for maximum likelihood estimation of a conditionally
Poisson model via data augmentation~\cite{Cemgil:2009}.

Placing a Gamma prior on the user weights gives the GaP
model~\cite{Canny:2004}, which was developed as an alternative text
model to latent Dirichlet allocation (LDA)~\cite{Blei:2003b}.

A difference between our treatment and GaP is that GaP fits the item
weights with maximum likelihood via the expectation maximization
algorithm.  Our model places Gamma priors on these weights (step 2,
above) and we approximate the full posterior with variational
inference.  Placing priors on both sets of weights further regularizes
the model and lets us use the same inferential machinery in both
user-space and item-space.  

% prem: removed reference to stochastic inference
%       we should add this in future work/discussion
%% Furthermore, using variational inference opens the door to scaling to
%% massive data sets, even larger than the data sets we analyze in this
%% paper, using stochastic variational inference~\cite{Hoffman:2013} (see
%% \mysec{inference}).

We note that GaP was developed as an alternative to LDA. In Appendix
A, we show that LDA can be reinterpreted as an instance of Poisson
factorization where we condition on the user counts and use an
alternative prior on the item weights.  (This connection was
previously unknown.)

Independently of GaP, Bayesian Poisson factorization has been studied
in the signal processing community for performing source separation
from spectrogram data~\cite{Cemgil:2009,Hoffman:2012}.  This research
includes variational approximations to the posterior, though the
issues and details around spectrogram data differ significantly from
user behavior data we consider and our derivation below (based on
auxiliary variables) is more direct.  As future work, the methods
developed here could lead to improved methods for massive simultaneous
analysis of audio spectrograms. In the context of network data, a
Poisson model of overlapping communities was described by the authors
of ~\cite{Ball:2011}. As with GaP, this model is unregularized and the
authors fit the model with maximum likelihood via expectation
maximization.

We discuss further differences between our method and previous work on
matrix factorization in the empirical results below.

% !!! add reference to canny's click data
% !!! add reference to SIGIR
XXX \cite{Marlin:2009,Marlin:2012,Elkan:2008,Ma:2011}
